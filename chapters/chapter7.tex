\glsresetall
\chapter{Conclusions and Future Work} \label{ch:con} % Add a label in case you want to refer to the chapter.
\section{Conclusions}
This thesis investigates an emerging research area to reconstruct ECG signals from the measured radar signals, and multiple contributions are made to improve different stages in radar-based ECG recovery:
\begin{itemize}
	\item Chapter~\ref{ch:cft} explores methods for efficiently collecting high-SNR radar signals that contain rich cardiac features, aiming to support deep-learning-based ECG recovery in later chapters. A novel \textbf{\gls{cft}} algorithm is designed to dynamically identify points with optimal SNR and track the cardiac location as subjects change postures.
	\item Chapter~\ref{ch:radarODE} aims to bridge the gap to realize a robust transformation from the mechanical domain to the electrical domain by proposing the signal model with fine-grained features considered. Furthermore, a deep learning framework \textbf{radarODE} is designed with morphological prior embedding as ODEs to provide faithful single-cycle ECG recoveries even under strong noises.
	\item Chapter~\ref{ch:radarMTL} is based on the robust single-cycle ECG generator designed in the previous chapter and further investigates the long-term ECG recovery under abrupt or constant noises. The realization of ECG recovery is appropriately deconstructed into three sub-tasks with an MTL framework called \textbf{radarODE-MTL} designed to generate long-term ECG signal. Additionally, a novel optimization strategy named EGA is presented to optimize all tasks simultaneously, effectively avoiding issues such as stalling or negative transfer. 
	\item Chapter~\ref{ch:RFcardi} tries to alleviate the data scarcity for DNN training, because radar-based ECG reconstruction is highly reliant on data-driven approaches. Firstly,  a data augmentation method called \textbf{Horcrux} is proposed to expand the diversity of the limited training dataset without distorting the key features. Secondly, a transfer learning framework called \textbf{RFcardi} is designed by leveraging an appropriate pre-text task (i.e., \gls{ssr}), enabling an effective learning of the latent representations from radar signals to assist the final ECG recovery task.
\end{itemize}


\section{Future Work}
The proposed \textbf{CFT}, \textbf{radarODE}, \textbf{radarODE-MTL}, \textbf{Horcrux} and \textbf{RFcardi} have shown outstanding performance compared with the previous work to generate faithful ECG signals under noisy scenarios using limited data, while the potential limitation will be discussed in this part to encourage future improvements in radar-based ECG recovery for real-life situations and applications.

\begin{itemize}
	\item \textbf{Long-range ECG Monitoring}: The direct impact brought by long-range monitoring is to reduce the SNR of the received radar signal according to the link budget analysis~\cite{zhang2024radarODE}. In addition, the cardiac location requires to be pre-identified to perform the accurate beamforming, because the current dataset assumes the majority of range-bins contain cardiac-related signals~\cite{chen2022contactless}.
	\item \textbf{Theoretical Model for Excitation-contraction Coupling}: The signal model (\ref{equ:vib}) and (\ref{equ:vib_long}) proposed in this thesis partially explain the recovering process from radar signal to ECG signal, while a key phase (i.e., excitation-contraction coupling~\cite{swift2021stop,orkand1964heart}) still needs a compact model because the current decoupling still purely relies on deep learning model, restring the reliable application especially for the implementation in clinical diagnosis.
	\item \textbf{Complex Monitoring Scenarios}: Various new noises might be introduced and need to be eliminated, such as radar self-vibration introduced from car vibrations or hand-held radar~\cite{da2019theoretical}, mutual-radar interference for the future smart home with multiple electromagnetic devices~\cite{yang2024isense} and signal attenuation caused by human tissues for the monitoring of people with random body orientations~\cite{liu2024diversity}.
	\item \textbf{Evaluation on the Dataset for Patients}: An important future application of radar-based ECG recovery is for clinical monitoring and diagnosis, while the ECG waveform for patients (e.g., arrhythmia) might be quite different, requiring massive new data for training. Some recent research has shown the feasibility of recovering abnormal ECG from radar signal~\cite{zhao2024airecg}, but more studies are required to investigate transfer learning or data augmentation techniques due to the scarcity of patient data. In addition, it is hard to preserve the noise-robustness for the ECG monitoring of patients, because the ODE model in this work is not designed for abnormal ECG patterns.
\end{itemize}

In summary, the robust radar-based ECG recovery still needs improvements from both theoretical and practical perspectives. The aforementioned research directions are still waiting for a thorough investigation to enable a realistic radar-based ECG measurement in our daily lives.



