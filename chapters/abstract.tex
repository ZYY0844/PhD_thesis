\chapter*{Abstract}
% \thispagestyle{fancy} 
Wireless sensing brings revolutions for multiple traditional industries and empowers numerous emerging industries such as autonomous driving and smart wellness monitoring. Among all the wireless sensors (e.g., Wi-Fi, acoustic sensors), radar-based sensing reveals outstanding performance in terms of detection range, applicable scenarios and robustness. This thesis will focus on a promising and challenging research area to realize the robust electrocardiogram (ECG) monitoring from millimeter-wave (mmWave) radar, enabling contactless vital sign monitoring for future in-cabin monitoring, elderly people caregiving and even clinical diagnosis.

Benefit from decades of development in radar-based coarse vital sign (e.g., heart rate, respiration) monitoring, frequency-modulated continuous wave (FMCW) radar with high operating frequency (e.g., $66$ or $77$Ghz) is becoming mainstream in radar front-end design, because such configurations could capture subtle displacements caused by cardiac activities and provide enough resolution for isolating targets from specific 3D points, encouraging the related research to extract fine-grained ECG signals as the golden standard in clinical diagnosis and realize robust monitoring in the presence of real-world noises.

In this thesis, radar-based ECG recovery is thoroughly investigated to provide a contactless ECG measurement that gets rid of cumbersome wired connections and adhesive electrode patches. However, several challenges need to be solved: (a) an efficient signal processing algorithm is required to ensure a high signal-to-noise ratio (SNR) for the collected radar signal, providing enough cardiac features for the later ECG recovery; (b) the domain transformation for the cardiac activities within single-cardiac cycle from mechanical to electrical domain needs to be modeled to realize a robust ECG recovery against noises; (c) the long-term ECG recovery should be modeled and realized instead of relying on the purely data-driven methods that can hardly resist noise disturbance; (d) the developed deep learning model normally relies on the large-scale radar/ECG dataset for training, and alleviating data scarcity is an important issues especially for the deployment in the new scenarios with limited data.

Based on the challenges above, multiple novel algorithms and deep learning frameworks are proposed. Firstly, a \textbf{cardio-focusing and -tracking (CFT)} algorithm is proposed to iteratively approach the point with a high-SNR radar signal extracted. Secondly, two deep learning models, \textbf{radarODE} and \textbf{radarODE-MTL}, are designed to realize robust single-cycle and long-term ECG recovery, respectively. Thirdly, a data augmentation method \textbf{Horcrux} and a transfer learning framework \textbf{RFcardi} are proposed to jointly decrease the demand for data acquisition. Finally, extensive experiments are performed based on both public and private datasets to show the effectiveness of the proposed algorithms and frameworks. It is believed that this thesis contributes to the general development of the wireless sensing community and brings the future applications of wireless wellness monitoring closer to our daily lives.