\glsresetall
\chapter{Introduction}\label{ch:intro}
The World Health Organization reports that the global population is aging at an unprecedented rate, with one in six people globally being over 60 by 2030, and the number of elderly people over $60$ will double to $2.1$ billion by 2050. This demographic shift brings significant challenges, including a higher prevalence of chronic diseases, frailty, and complex health conditions among older adults~\cite{WHO2024ageing}. Maintaining health and independence in later life is crucial not just for individuals, but for families and societies. As the world population ages, the early detection of diseases and daily monitoring of health conditions become critical to reducing the need for institutional care and improving quality of life, but reaching assistance from professional caregivers might be expensive or inconvenient especially when most of the older population will live in low- and middle-income countries~\cite{WHO2024ageing}. Therefore, the current labor-intensive wellness monitoring needs a revolution, and wireless vital sign monitoring addresses these challenges by enabling continuous, automated monitoring of health indicators or emergency events (e.g., heart rate and fall detection) without constant human intervention.

\section{Wireless Sensing for Vital Sign Monitoring}
In the recent decade, wireless sensing has emerged as a transformative paradigm that leverages electromagnetic waves to remotely detect changes and gather information about objects, environments, and biological entities, and this non-invasive approach has been embedded in a wide range of applications from healthcare monitoring to autonomous driving, leading the rapid evolution of information and communication technologies towards the next generation~\cite{zhang2023overview,guan2023achelous,wang2020vimo}. Different from traditional contact-based sensing methods that require physical connections to the subject or environment, wireless sensing can be seamlessly integrated into daily life, fostering innovation in smart infrastructure and personalized health systems. For example, by leveraging radio frequency signals to detect subtle physiological movements such as respiration and heartbeat, radar-based systems eliminate the need for physical contact while maintaining clinical-grade accuracy, addressing key limitations of traditional wired sensors that cause patient discomfort, skin irritation, and infection risks~\cite{zhang2023overview}.

Various wireless sensors can be utilized for vital sign monitoring according to the information received. For example, optical or thermal cameras sense the skin color or temperature variation caused by heartbeats~\cite{chen2018video,chian2022vital}, acoustic sensors can monitor the heart sound~\cite{xu2020leveraging}, and Wi-Fi routers extract cardiac features from the channel state information~\cite{jia2018wifind}. However, the aforementioned sensors may be blamed for privacy issue~\cite{feng2021multitarget}, low accuracy~\cite{nirmal2021deep} or vulnerability to the changing environment (e.g., light conditions or temperature variations)~\cite{xia2021radar}. In contrast, radar senses the ambient environment through reflected signals mixed by chest wall displacement (induced by respiratory and cardiac activities) and all kinds of ambient noises~\cite{wang2020remote}, requiring proper algorithms to further extract the latent vital features. Additionally, compared with cameras, Wi-Fi routers and acoustic sensors, radar signal propagation is neither vulnerable to illumination/temperature/sound variations nor privacy-intrusive. Therefore, radar-based vital sign monitoring is promising to realize unobtrusive sensing in most scenarios after the design of advanced signal processing algorithms.

\section{Radar-based Vital Sign Monitoring}
\begin{figure}[tb] 
    \centering 
    \includegraphics[width=0.9\columnwidth]{intro.pdf}
    \caption{Illustration of different steps in radar-based cardiac monitoring.}
    \label{fig:intro_overall} 
\end{figure}
The first attempt at radar-based vital sign monitoring can be traced back to 1975 by measuring the displacement of the chest wall induced by respiration~\cite{lin1975noninvasive}. The chest wall displacement will modulate the phase component of the emitted radar signal, and the latent respiratory information can be demodulated from the phase variation~\cite{wang2020remote}. Similarly, cardiac activities are small-scale displacements that also cause chest wall displacements, but such small displacements are normally ruined by respiration with orders more amplitude and need to be extracted using advanced signal processing algorithms. 

The overview of the radar-based cardiac monitoring process and the downstream applications are shown in Figure~\ref{fig:intro_overall}. To measure the target cardiac features, the first step is to transmit the radio frequency signals with certain waveforms to sense the ambient environment as illustrated in Figure~\ref{fig:intro_overall}(b). After reflection, the received signals contain not only the target signal (vibration caused by heartbeats) but also other noise signals such as \gls{rbm}, respiration and multi-person or multi-path interferences, as shown in Figure~\ref{fig:intro_overall}(a). The second step aims to isolate the signal reflected by a single human body from the background clutter or the interference from neighbors using specific methods, as shown in Figure~\ref{fig:intro_overall}(c)~\cite{islam2022contactless}. The third step is to extract the cardiac features from the reflected signals of a single person using various algorithms, with detailed classification provided in Figure~\ref{fig:intro_overall}(d). The last step is to analyze the obtained cardiac features for specific downstream applications~\cite{Liu2022Integrated, Wang2023STARS, Liu2022Holographic, Liu2022Cramer, Hua2022Performance, Liu2022Generalized, Xu2022Target, Ren2022Joint, Liu2022Joint, Jiang2022STAR, Xu2021Simultaneously, Stoica2006Target, Liu2020Joint, Liu2022MetaRadar, Luo2022Cramer, Liu2021Enabling, Mu2021STAR, Wu2021Reconfigurable, Liu2018MU, Yu2021Channel}, as shown in Figure~\ref{fig:intro_overall}(e). 

Most early studies focused on the recovery of coarse cardiac information, such as \gls{hr}, heart sound and \gls{hrv}, from the perspectives of radar front-end design or advanced algorithms design~\cite{zhang2023overview}. For example, some advanced types of radar (e.g., \gls{fmcw} radar) are designed to enable high range-resolution or multi-person monitoring~\cite{ha2020contactless}, and some baseband signal processing algorithms are embedded on the radar platform to realize \gls{iq} modulation or accurate phase unwrapping~\cite{obadi2021survey}. In addition, various advanced algorithms are applied by leveraging different intrinsic characteristics of cardiac activities to robustly reconstruct cardiac features. For example, cardiac activities normally reveal strong periodicity in the time domain and have dominant peaks on the spectrum, inspiring periodicity-based methods (e.g., template matching~\cite{lv2018doppler}, hidden Markov model~\cite{xia2021radar}) and spectrum-based methods (e.g., Fourier transform~\cite{nosrati2018accurate}, wavelet transform~\cite{mercuri2019vital}) as two major categories in cardiac feature extraction algorithms.

In recent years, the emergence of commercial radar platforms with high operating frequency (i.e., \gls{mmWave} radar) encourages researchers to extract fine-grained cardiac features (e.g., \gls{ecg} and \gls{scg}) from the radar signal~\cite{zhang2023overview}. SCG signal is measured by the accelerometer mounted on the human chest to measure the mechanical vibrations produced by heartbeats, describing the fine-grained cardiac mechanical activities such as aortic/mitral valve opening/closing and isovolumetric contraction~\cite{cocconcelli2020high}. Although these vibrations are subtle, it is still reasonable to directly map the displacements detected by radar to each fine-grained cardiac mechanical activity using high-resolution radar as proved in~\cite{ha2020contactless}. However, the SCG measurement is not widely used in clinical scenarios, while the ECG signal is commonly recognized as the golden standard in cardiac monitoring because ECG describes the fine-grained cardiac activities, such as atrial/ventricular depolarization/repolarization, through the featured waveform (i.e., PQRST peaks) and is crucial to the diagnosis of cardiovascular diseases~\cite{swift2021stop}.

Based on the discussions above, it is believed that radar-based ECG recovery is an essential research direction to enable the realization of future wellness monitoring, even providing fine-grained cardiac features for clinical diagnosis. This thesis will thoroughly investigate all the stages in radar-based ECG recovery with multiply inventions proposed to overcome the unsolved challenges. In addition, this research has been approved by the University Ethics Committee of Xi'an Jiaotong-Liverpool University with proposal number ER-SAT-0010000090020220906151929.

\section{Challenges in Radar-based ECG Recovery}
The realization of radar-based ECG recovery needs two general stages: (a) radar signal collection and pre-processing to increase \gls{snr}; (b) ECG recovery from radar signal to realize domain transformation from cardiac mechanical activities to electrical activities using \gls{dnn}, and this thesis will elaborate on four challenges figured in both stages with corresponding solutions proposed as shown in Figure~\ref{fig:first_plot}.

\begin{figure}[tbp] 
    \centering 
    \includegraphics[width=0.9\columnwidth]{first_plot.pdf}
    \caption{Overview of the thesis with unsolved challenges.}
    \label{fig:first_plot} 
\end{figure}

\subsection{Efficient High-SNR Signal Acquisition}\label{sec:the_first_challenge}
In the literature, many studies are dedicated to inventing advanced signal processing algorithms to enhance the signal quality, because the deep learning model for ECG recovery is vulnerable to the inputs contaminated by noises and requires high-SNR radar signals as inputs~\cite{chen2022contactless,liu2024diversity}. The methods for capturing high-SNR radar signals can be categorized into two groups:
\begin{itemize}
\item The first type of method focuses on designing advanced radar front-end with multiple \gls{Txs} and \gls{Rxs}~\cite{li2024robust,xiong2022vital} or calibrating baseband radar signals from in-phase and quadrature (IQ) channels to a circular shape~\cite{dong2024robust,ni2024accurate,zhang2024single}.
\item The second type of method assumes that the rough localization of human body provides accurate chest region with the majority of range bins containing useful cardiac features, and high-SNR signal can be obtained by selecting useful channels~\cite{li2024radarnet,zhang2025umimo}, applying clustering algorithms~\cite{chen2022contactless} or accumulating the signals from various dimensions (e.g., chirps, frames, antennas)~\cite{liu2024diversity}. 
\end{itemize}

The first type of method is not suitable for some commonly used frequency-modulated continuous-wave (FMCW) radar platforms (e.g., TI AWR-x radar) due to the on-board digital front-end module filtering the frequency-modulated feature of baseband signal (i.e., circular IQ plot)~\cite{chen2024co}, preventing the broad applications of this approach in commercial radar. The second type of method relies on accurate localization of the chest region, while the existing methods only provide a rough location of the human body, causing a deviation of several decimeters due to different postures of the subject~\cite{chen2021movi}. Therefore, the methods based on signal accumulation may fail because only a minority of range bins contain cardiac features, hence not subjecting to the law of large numbers~\cite{liu2024diversity}. Although some aforementioned studies have proposed methods for selecting or clustering the useful range bins with cardiac features~\cite{chen2022contactless,li2024radarnet}, the computational cost for traversing a large objective space can be huge without an accurate cardiac location. In this case, the \textbf{first challenge} is to precisely locate and track the cardiac location during data collection to efficiently extract high-SNR radar signal.

\subsection{Robust Single-cycle ECG Generation}\label{sec:the_second_challenge}
To reconstruct ECG from radar signal, the researchers must deal with domain decoupling to transform the measured signal from the mechanical domain to the electrical domain to generate ECG measurement. Intuitively, it is reasonable that mechanical conduction and electrical conduction are highly correlated in describing cardiac activities because the electrical changes in cells trigger heart muscle contraction, whereas such a relationship is called excitation-contraction coupling in electrophysiology and is extremely hard to interpret or model by researchers without biological knowledge~\cite{swift2021stop,orkand1964heart}.

In the literature, the existing studies all leverage deep learning methods to extract latent information from enormous radar/ECG pairs and try to learn domain transformation relying on the extraordinary non-linear mapping ability of the deep neural network~\cite{chen2022contactless,wu2023contactless,li2024radarnet,wang2023ecg,zhao2024airecg}. Although these studies could successfully reconstruct the ECG signal from radar, there is no existing signal model with a compact form to describe the domain transformation for radar-based ECG reconstruction, and the well-trained deep learning model is not robust to abrupt noises such as RMB~\cite{chen2022contactless,wang2023vital}, because these noises normally have orders of magnitude higher than cardiac-related vibrations, drowning out subtle features and ruining forward propagation of the deep neural network~\cite{wang2023slprof}. Therefore, the \textbf{second challenge} is to design the signal model
that considers fine-grained cardiac features within a single cardiac cycle and designs a robust single-cycle ECG generation module against abrupt noises.

\subsection{Robust Long-term ECG Generation}\label{sec:the_third_challenge}
After the modeling and recovery of single-cycle ECG pieces, a follow-up issue is to model and realize the robust long-term ECG recovery. Although the model for the domain transformation between single-cycle radar/ECG pair has been proposed in~\cite{zhang2024radarODE}, the long-term ECG recovery might be misaligned with ground truth due to inaccurate \gls{ppi} estimation, deteriorating the recovery quality even if the morphological features are well-recovered. Therefore, the \textbf{third challenge} is to model and generate the long-term ECG recovery from radar signal, and the recovery process should be robust against noises.

\subsection{Alleviation of Data Scarcity}\label{sec:the_fourth_challenge}
According to the literature, radar-based ECG recovery is only realized by deep-learning-based methods, because the domain transformation is extremely complex to be modeled mathematically while such transformation can be learned by deep learning model due to the great nonlinear mapping ability~\cite{chen2022contactless}. Similar to other research fields involved with deep learning, radar-based ECG recovery also asks for numerous radar signals to train the deep learning model with synchronous ECG ground truths~\cite{li2024radarnet,zhang2024radarODE-MTL,chu2024vessel}. According to previous research, the performance of the deep learning model degrades heavily after reducing $30\%$ of the training data even after applying proper data augmentation techniques~\cite{zhang2025horcrux}, causing difficulties for the deployment in new scenarios due to the demand for hours of ECG collection~\cite{zhang2024radarODE}. However, the method for reducing dependence on data quantity is rarely investigated for radar-based ECG recovery, and all the deep-learning-based ECG recovery models are trained in a supervised manner with large dataset containing $3-32$ hours of synchronous radar-ECG pairs~\cite{chen2022contactless,zhao2024airecg,li2024radarnet}. In this case, the \textbf{fourth challenge} is to reduce the dependency on large-scale datasets and develop appropriate transfer learning or data augmentation methods to alleviate data scarcity, especially for the deployment in new scenarios with limited data.

\section{Contributions and Thesis Outline}
The proposed solutions for the aforementioned challenges are elaborated in Chapter~\ref{ch:cft},~\ref{ch:radarODE},~\ref{ch:radarMTL} and~\ref{ch:RFcardi} with a general outline shown in Figure~\ref{fig:first_plot}, and the main contributions of this thesis can be concluded as:
\begin{itemize}
  \item Chapter~\ref{ch:cft} proposes a \gls{cft} algorithm based on \gls{dfo} to find the \gls{cf} point by iteratively evaluating the potential points in a discontinuous objective space, with a universal signal template designed to adaptively assess the signal SNR as costs.
  \item Chapter~\ref{ch:radarODE} designs a signal model to describe the fine-grained cardiac feature sensed by radar, enabling further domain transformation between cardiac mechanical and electrical activities. Based on the proposed signal model, an ODE-embedded module called \gls{sceg} is designed to realize the domain transformation by parameterizing the radar signal into sparse representations and hence generate the morphological ECG features as references to resist noise.
  \item Chapter~\ref{ch:radarMTL} investigates the noise robustness in radar-based ECG recovery against constant or abrupt noise by modeling the cardiac domain transformation as three tasks. Then, an end-to-end \gls{mtl} framework named radarODE-MTL is accordingly proposed to realize these tasks and leverage adjacent cardiac cycles to compensate for the distorted one. To assist the MTL model training, a novel optimization strategy called \gls{ega} is proposed for updating shared parameters in the MTL neural network, aiming to balance the intrinsic difficulty across tasks during network training and also prevent the negative transfer phenomenon. 
  \item Chapter~\ref{ch:RFcardi} tries to alleviate data scarcity in deep learning model training from two perspectives. Firstly, a transfer learning framework RFcardi is proposed following a \gls{ssl} paradigm to effectively learn the latent representations from radar signals by leveraging an appropriate pre-text task. Secondly, a data augmentation method called Horcrux is proposed to expand the diversity of the limited training dataset without distorting the intrinsic time consistency hidden in the radar inputs.
\end{itemize}

In addition, Chapter~\ref{ch:LR} provides the necessary background for radar-based vital sign monitoring, and Chapter~\ref{ch:con} concludes the entire thesis with several promising research directions for future investigation.

